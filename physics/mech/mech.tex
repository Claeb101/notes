\documentclass{article}
\usepackage[utf8]{inputenc}

\title{Lecture Notes for Advanced Math Techniques}
\author{Alvan Caleb Arulandu}
\date{January 2021}

\usepackage{amssymb, amsmath, amsfonts}
\usepackage{graphicx}

\begin{document}

\maketitle

\section{Kinematics}
    Kinematics lies a multiple fundemantal definitions regarding velocity, acceleration, and position. 
    Additionally, there are a number of helpful quick formulas that are nice to memorize but easy to derive. Note
    that motion in different coordinate directions are independent.
    $$h_{max}=\frac{v_{0}^{2}\sin^{2}\theta}{2g};\, t_{top}=\frac{v_{0}\cos\theta}{g};\, d_{range}=\frac{v_{0}\sin 2\theta}{g}$$
\section{Reference Frames}
    \subsection{Inertial Frames}
        An \textbf{intertial frame} is a frame of reference that is not accelerating or rotating.
        If A is an intertial frame, and B is moving at a constant velocity with respect to A, then B's frame is also inertial.
        Note that if 2 objects are rotating relative to each other in space, only one can be inertial.

    \subsection{Rotated Frames}
        Rotating the frame of reference is very useful and can simplify many calculations.

\section{Collisions}
    Remember to account for frictional impulse during collisions. This stems from the normal impulse that slows an object to rest. See Problem 3.2 in {\it Relatvitiy and Collsion Handout PhysicsWOOT 20-21}.
    $$j_{f}\leq \mu j_{N}$$
    In 1D elastic collisions, the relative velocity between the objects changes sign, so
    $$m_{1}v_{1i}+m_{2}v_{2i}=m_{1}v_{1f}+m_{2}v_{2f}$$
    $$v_{1i}-v_{2i}=v_{2f}-v_{1f}$$
    If the masses are equal, the objects simply switch velocities.

    \subsection{Elastic Collisions}
        Elastic collisions conserve energy and happen in a "perfect world".

    \subsection{Inelastic Collisions}
        This is when energy is not conserved but is dissapatted as heat and other forms of energy.

    \subsection{Perfectly Inelastic Collisions}
        These collisions disapate the most kinetic energy and causes the colliding objects to

\section{Differential Equations}
    \section{Sinusoidal Motion}
        The following differential equation results in sinusoidal motion with angular velocity $\omega$.
        $$\frac{d^{2}x}{dt^{2}}=-\omega^{2}x$$

        The potential energy function is parabolic.
        $$U(x)=ax^{2}$$

        The solution to this is a function of the following form,
        $$x(t)=\frac{v_{0}}{\omega}\sin(\omega (x-x_{0})+\phi)+x_{0}$$
        Note that $A=\frac{v_{0}}{\omega}$, $\phi$ is the angular offset and $x_{0}$ is the "middle" value.

\section{Forces}
    The normal force resists compression. The tension force resists stretching. The frictional force can apply in
    different directions (static friction).

\section{Random}
    \subsection{Morse Potential}
        In chemistry, double bonds can be thought of as a spring with variable bond length that vibrates back
        and forth. The potential energy in the spring is known as the morse potential.

\section{Statics}
    Statics are when both torque and forces sum to 0. Draw the forces in properly and quickly balance using F=ma. Cancel out the horizontal and vertical forces to see
    if that gives any easy substitutions. Then, try picking a pivot point and analyzing the torques. Pick a pivot with
    a lot of forces on it so that all their torques are zero. For some connected problems, try analyzing the torque of 
    the entire system. Remember that pivots can exert forces in the "frictional" direction and into the wall as well. 
    Remember that at pivots, Newton's 3rd law applies so if object 1 exerts a force on object 2, an opposite force is 
    exerted on object 1.

\section{Gravity}
    Newtons law states that the force due to gravity is 
    $$\vec{F}=\frac{Gm_{1}m_{2}}{R^{2}}\hat{r}$$
    Then, 
    $$U=-\int_{0}^{\infty}Fdr=-\frac{Gm_{1}m_{2}}{R}$$
    Note that this is with respect to $\infty$ at which the potential energy is 0.
    The main idea is that potential energy should decrease as objects get closer together.

\section{Orbits}
    Orbits conserve angular momentum. From centripetal acceleration it is evident that,
    $$v_{circ}=\sqrt{\frac{GM}{R}};\, E=-K=\frac{U}{2};$$
    We can also use conservation of energy to find the escape speed.
    $$v_{esc}=\sqrt{\frac{2GM}{R}}$$
    Using this we can classify orbits by tangential velocity. At \textbf{apogee}, the object is at a maximum distance away.
    Note that if $0<v_{tan}<v_{circ}$ then the orbit must be elliptical and $R$ is the max distance from earth (apogee).
    If $v_{tan}=v_{circ}$, the orbit is circular with radius $R$. If $v_{circ}<v_{tan}<v_{esc}$, the orbit is elliptical and 
    $R$ is the minimum distance from Earth (perigee). If $v_{tan}=v_{esc}$, the object escapes the orbit parabolically.
    But if $v_{tan}>v_{esc}$, then the object exits the orbit but hyperbolically.

\section{Fluids}
    \textbf{Archimedes Principle} states that the force that a fluid exerts on an object equals the weight
    of the water it displaces or
    $$F_{bouyant}=m_{disp}g=\rho V_{disp}g$$
    \textbf{Pascal's Principle} states that when a pressure is applied to a confined fluid, the fluid exerts
    that pressure everywhere. This can be applied to pistons showing that
    $$\frac{F_{1}}{A_{1}}=\frac{F_{2}}{A_{2}}$$
    \textbf{Burnoulli's Principle} is simply conservation of energy applied to fluids. 
    $$P+\frac{1}{2}\rho v^{2} + \rho gh = C\in \mathbb{R}$$
    When you multiply both sides by V, we see that this is simply energy required to apply a pressure on the 
    fluid volume in "front" plus the kinetic energy due to velocity plus the gravitational potential energy.
    Dividing by volume allows us to to not worry about volume but instead density which is easier to deal with.
    Remember that the pressure energy is the atmospheric pressure when exposed to air.

\section{Rotation}
    When you have a ball that is rolling and moving, the center of mass moves with velocity v, but the point 
    on the ball has to rotate AND move meaning it has velocity v + v = 2v (second v from centripetal motion).
    \textbf{Roling without slipping} implies that $v=rw$. If an object takes time to roll, find the time it takes
    for that equation to be true. Note that all points on the boundary do \textbf{not} rotate with the same speed (break-up components).
    The Earth rotates counter clockwise when looking down at the north pole.

\section{Ficticious Forces}
    \textbf{Translational} forces are if you reference frame is linearly accelerating. Add a translational force
    in the opposite direction.
    $$F_{trans}=-\frac{d^{2}\vec{R}}{dt^{2}}$$

    \textbf{Centrifugal} forces should be added if you have a rotating reference frame with a static object. Then 
    add the centrifugal force in the opposite direction of the "typical" centripital force in the outside reference
    frame. 
    $$F_{cf}=-m\omega\times(\omega\times r)$$

    \textbf{Coriollis} forces are added if you have a rotating reference frame and the object has a velocity. Then, 
    the coriollis force is
    $$F_{cor}=-2m(\vec{\omega}\times \vec{v})$$

    \textbf{Azimuthal} forces are added when the reference frame itself is rotating with an angular acceleration. The azimuthal
    force is in the opposite direction of a typical tangential acceleration.
    $$F_{az}=-m\alpha\times r$$
    Remember that during centripetal motion, there vector $\vec{r}$ points from the center along the radius to the arc/path.

\section{Graphs}
    Check limiting cases especially on graph questions. Often times, you don't even need
    to solve completely. You just need to take some limits and maybe derivatives.

\section{Uncertainty}
    Remember that quadrature is a Guassian approximation. Don't jump into quadrature right away but see if you can use a mathematical
    exploit with a log or something to make relative uncertainty easier to work with for example. Note that we don't always have to use 
    quadrature. Try plugging in some delta's to figure changes.

\section{Power}
    Power is the integral of force with respect to velocity or the derivative of work with respect to time. Note that this means power 
    is proportional to energy.

\section{Work}
    For work, remember the linear and rotational analog $W=Fd$ and $W=\tau\theta$. Work is related to energy via the work-kinetic energy
    theorem.

\end{document}
 
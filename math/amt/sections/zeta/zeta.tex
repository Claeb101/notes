\documentclass[../main.tex]{subfiles}
\begin{document}
\chapter{The Zeta Function}

\section{The $\sin$ function, the Basel Problem, and more results}
    \subsection{Beginning}
        $\sin(0)=0$. So, let's divide by $x$ to remove the factor of $x$ in the expansion.
        $$\frac{\sin(x)}{x}=(1-\frac{x}{\pi})(1+\frac{x}{\pi})(1-\frac{x}{2\pi})(1+\frac{x}{2\pi})\cdots$$
        Using difference of squares,
        $$=(1-\frac{x^{2}}{\pi^{2}})(1-\frac{x^{2}}{4\pi^{2}})(1-\frac{x^{2}}{9\pi^{2}})\cdots$$
        Note that we have not established these 2 functions are equivalent but that they have the same zeroes. This was used to say Euler's argument wasn't
        rigorous (the full thing took another 10 years). Without the proven rigour, let's suppose this statement.
        Expanding the product by powers of $x$,
        $$=1-\frac{x^{2}}{\pi^{2}}(1+\frac{1}{4}+\frac{1}{9}+\cdots)+\frac{x^{4}}{\pi^{4}}(\frac{1}{1\cdot 4}+\frac{1}{1\cdot 9} + \cdots + \frac{1}{4 \cdot 9} + \frac{1}{4 \cdot 16}+\cdots)-\cdots$$
        $$=1-\frac{x^{2}}{\pi^{2}}\sum_{n=1}^{\infty}\frac{1}{n^{2}}-\frac{x^{4}}{\pi^{4}}\sum_{n=1}^{\infty}\sum_{m=n+1}^{\infty}\frac{1}{n^{2}m^{2}}+\cdots$$
 
    \subsection{Deriving results}
        Using the taylor series,
        $$\frac{\sin(x)}{x}=1-\frac{x^{2}}{3!}+\frac{x^{4}}{5!}-\cdots$$
        Note that these 2 expansions are equal. This \textbf{solves the Basel problem}.
        $$\therefore \frac{-X^{2}}{3!}=\frac{-X^{2}}{\pi^{2}}\sum_{n=1}^{\infty}\frac{1}{n^{2}}\Rightarrow \sum_{n=1}^{\infty}\frac{1}{n^{2}}=\frac{\pi^{2}}{6}$$
        We can use this for other series.
        $$\left(\sum_{n=1}^{\infty}\frac{1}{n^{2}}\right)\left(\sum_{m=1}^{\infty}\frac{1}{m^{2}}\right)=\frac{\pi^{4}}{36}$$
        Also,
        $$=\sum_{n=1}^{\infty}\frac{1}{n^{4}}+\sum_{n=1}^{\infty}\sum_{m=n+1}^{\infty}\frac{1}{m^{2}n^{2}}+\sum_{n=1}^{\infty}\sum_{m=n+1}^{\infty}\frac{1}{m^{2}n^{2}}$$
        $$\therefore \frac{\pi^{4}}{36}=\sum_{n=1}^{\infty}+\frac{2\pi^{4}}{120}\Rightarrow \sum_{n=1}^{\infty}\frac{1}{n^{4}}=\frac{\pi^{4}}{90}$$
        Euler continued this all the way to $\sum_{n=1}^{\infty}\frac{1}{n^{26}}$.

    \subsection{Conversion}
        To write it in a closed form,
        $$\frac{sin(x)}{x}=\prod_{k=1}^{\infty}(1-\frac{x^{2}}{k^{2}\pi^{2}})$$
        Writing as a series,
        $$\ln\frac{sin(x)}{x}=\sum_{k=1}^{\infty}ln\left(1-\frac{x^{2}}{k^{2}+\pi^{2}}\right)$$
        $$=-\sum_{k=1}^{\infty}\sum_{j=1}\frac{\frac{x^{2}}{k^{2}\pi^{2}}}{j}\Rightarrow -sum_{j=1}^{\infty}\frac{x^{2j}}{j}\frac{\zeta(2j)}{\pi^{2j}}$$
        Looking at cotangent,
        $$\cot(x)-\frac{1}{x}=-\sum_{j=1}^{\infty}2\frac{\zeta(2j)}{\pi^{2}j}x^{2j-1}$$

\section{Dirichlet Series}
    \subsection{Zeta Series}
        This function is not really a dirichlet series but it's related.
        $$\zeta(s)=\sum_{n=1}^{\infty}\frac{1}{n^{s}}=1+\frac{1}{2^{s}}+\frac{1}{3^{s}}+\cdots$$
    \subsection{Eta Series}
        This series is an alternating Zeta Series.
        $$\eta(s)=\sum_{n=1}^{\infty}\frac{(-1)^{n+1}}{n^{2}}=1-\frac{1}{2^{s}}+\frac{1}{3^{s}}-\cdots$$
    \subsection{Lambda Series}
        This series is a Zeta Series with only odd terms.
        $$\lambda(s)=\sum_{k=0}^{\infty}\frac{1}{(2k+1)^{s}}=1+\frac{1}{3^{s}}+\frac{1}{5^{s}}+\cdots$$
    \subsection{Beta Series}
        This series is an alternating lambda series.
        $$\lambda(s)=\sum_{k=0}^{\infty}\frac{(-1)^{k}}{(2k+1)^{s}}=1-\frac{1}{3^{s}}+\frac{1}{5^{s}}-\cdots$$
    \subsection{Even Zeroes}
        Eueler has given all the even zeroes of the zeta function.
        $$\zeta(2)=\frac{\pi^{2}}{6}$$
        $$\zeta(4)=\frac{\pi^{4}}{90}$$
        $$\zeta(6)=\frac{\pi^{6}}{945}$$
    \subsection{Deriving Additional Values}
        $$\eta(2)=\frac{\pi^{2}}{12}$$
        $$\lambda(2)=\frac{3\pi^{2}}{24}$$

    \subsection{Deriving Additional Results}
        $$\eta(s)=\zeta(s)-2(\frac{1}{2^{s}}\zeta(s))=(1-2^{1-s})\zeta(s)$$
        $$\lambda(s)=\zeta(s)-(\frac{1}{2^{s}}\zeta(s))=(1-2^{-s})\zeta(s)$$
        $\beta(s)$ is not related to the other functions.

    \subsection{Apery's Constant}
        Apery's constant is $\zeta(3)$ because the French mathematician proved that 
        it was irrational. For zeta, the odd's are hard and the even's are known exactly.

    \subsection{Catalan's Constant}
        Catalan's constant is $\beta(2)$. For $\beta$ the even's are hard and the odd ones are known.
        
\section{Weirstrass Approximation Theorem}
    You can approximate an arbitrarily continuous function by an arbitrary polynomial.

\end{document}
\documentclass[../main.tex]{subfiles}
\begin{document}
\chapter{Exploring the Zeta Function}
\section{Legendre Duplication Identity}
    $$\int_{0}^{1}t^{z-1}(1-t)^{z-1}dt=\frac{\Gamma(z)\Gamma(z)}{\Gamma(2z)}$$
    Now, let $u=2t-1$. This allows more symmetry in integral bounds.
    $$=\int_{-1}^{1}\left(\frac{1+u}{2}\right)^{z-1}\left(\frac{1-u}{2}\right)^{z-1}\frac{du}{2}=\frac{1}{2^{2z-1}}\int_{-1}^{1}(1-u^{2})^{z-1}du$$
    By symmetry,
    $$=\frac{2}{2^2z-1}\int_{0}^{1}(1-u^2)^{z-1}du$$.
    Let $v=u^{2}$,
    $$=\frac{2}{2^{2z-1}}\int_{0}^{1}(1-v)^{z-1}\frac{1}{2}v^{-\frac{1}{2}}dv=\frac{1}{2z-1}\frac{\Gamma(z)\Gamma(\frac{1}{2})}{\Gamma(z+\frac{1}{2})}=$$
    Rewriting,
    $$\boxed{\Gamma(2z)=2^{2z-1}\frac{\Gamma(z)\Gamma(z+\frac{1}{2})}{\sqrt{\pi}}}$$
    This is the Legendre Duplication Identity.

\section{digamma?}
    We want to now shift $\Gamma$ by $\frac{1}{2}$.
    $$\Gamma(\frac{1}{2}+\epsilon)=\frac{\sqrt{\pi}}{2^{2\epsilon-1}}\frac{\Gamma(2\epsilon)}{\Gamma(\epsilon)}=\frac{\sqrt{\pi}}{2^{2\epsilon-1}}\frac{2\epsilon\Gamma(2\epsilon)}{2\epsilon\Gamma(\epsilon)}=2^{-2\epsilon}\sqrt{\pi}\frac{\Gamma(1+2\epsilon)}{\Gamma(1+\epsilon)}$$
    This shift allows us to generate many integrals using expansions like before (this time using Legendre's). Namely, $\int_{0}^{\infty}\sqrt{x}\ln^{2}x e^{-x}$. 

\section{Regulators}
    Regulators can be used to prevent the divergence of an integral "long enough" to extract its value. This can only be used if the integral converges.
    In essence, when an integral converges, sometimes the only way to use the gamma function is to split the integral up into two separately divergent integrals.
    Regulators allow us to continue with this without breaking convergence.
    Find $\int_{0}^{\frac{\pi}{2}}\frac{\ln\sin\theta}{\cos\theta}d\theta$. Adding a regulator and auxiliarly parameter, lets consider instead
    $$\int_{0}^{\frac{\pi}{2}}\cos^{-1+\delta}\theta\cdot\sin^{\epsilon}\theta d\theta=\frac{\Gamma(\frac{\delta}{2})\Gamma(\frac{1+\epsilon}{2})}{2\Gamma(1+\epsilon+\delta)}{2}=\frac{1}{\delta}\cdot 2^{\delta}\frac{\Gamma(1+\frac{\delta}{2})\Gamma(1+\epsilon)\Gamma(1+\frac{\epsilon+\delta}{2})}{\Gamma(1+\frac{\epsilon}{2})\Gamma(1+\epsilon+\delta)}$$
    Expanding and cancelling,
    $$=\frac{1}{\delta}e^{\delta\ln 2}\exp\left\{\sum_{k=2}^{\infty}\frac{(-1)^{k}\zeta(k)}{k}\left[\left(\frac{\delta}{2}\right)^{k}+\epsilon^{k}+\left(\frac{\epsilon+\delta}{2}\right)^{k}-\left(\frac{\epsilon}{2}\right)^{k}-(\epsilon+\delta)^{k}\right]\right\}$$
    $$=\frac{1}{\delta}e^{\delta\ln 2}\exp\left\{\frac{\zeta(2)}{2}-\frac{\delta^{2}}{2}-\frac{3}{2}\epsilon\delta+\cdots\right\}$$
    Isolating the $\epsilon\delta$ term as it is the only one that relates to the original integral,
    $$\Rightarrow\frac{1}{\delta}\left(-\frac{3}{4}\zeta(2)\epsilon\delta\right)\Rightarrow-\frac{\pi^{2}}{8}$$

\section{Integral for the Zeta Function}
    $$\int_{0}^{\infty}\frac{x^{n-1}}{e^{ax}-1}dx=\int_{0}^{\infty}\frac{x^{n-1}}{e^{ax}}\cdot\frac{1}{1-e^{-ax}}dx=\int_{0}^{\infty}x^{n-1}e^{-ax}\sum_{k=0}^{\infty}(e^{-ax})^{k}dx$$
    Changing the order of integration and summation (not rigorous but it will work),
    $$=\sum_{k=0}^{\infty}\int_{0}^{\infty}x^{n-1}e^{-(k+1)ax}dx=\sum_{k=0}^{\infty}\frac{1}{a^{n}(k+1)^{n}}\int{0}^{\infty}u^{n-1}e^{-u}du$$
    $$=a^{-n}\Gamma(n)\sum_{k=1}^{\infty}\frac{1}{k^{n}}=a^{-n}\Gamma(n)\zeta(n)$$
    Differentiating with respect to a,
    $$=\int_{0}^{\infty}\frac{x^{n}e^{ax}}{(e^{ax}-1)^{2}}=a^{-n-1}\Gamma(n+1)\zeta(n)$$
    Maniuplating to get another integral,
    $$a^{-n-1}\Gamma(n+1)\zeta(n)=\int_{0}^{\infty}\frac{x^{n}(e^{ax}-1+1)}{(e^{ax}-1)^{2}}$$
    $$=\int_{0}^{\infty}\frac{x^{n}}{(e^{ax}-1)}dx+\int_{0}^{\infty}\frac{x^{n}}{(e^{ax}-1)^{2}}dx$$
    $$\therefore \int_{0}^{\infty}\frac{x^{n}}{(e^{ax}-1)^{2}}=a^{-n-1}\Gamma(n+1)(\zeta(n)-\zeta(n+1))$$

\section{Product representation for zeta}
    We have said before that
    $$\lambda(s)=1+\frac{1}{3^{s}}+\frac{1}{5^{s}}+\cdots=\zeta(s)-\frac{1}{2^{s}}\zeta(s)=(1-\frac{1}{2^{s}})\zeta(s)$$
    This is found by subtracting all the even numbers. Let's continue but this time subtracting all the multiples of 3.
    $$(1-\frac{1}{3^{s}})(1-\frac{1}{2^{s}})\zeta(s)=1+\frac{1}{3^{s}}+\cdots-\left(\frac{1}{3^{s}}+\frac{1}{9^{2}}+\cdots\right)$$
    Continuing to do so with all of the primes,
    $$\zeta(s)\prod_{p\in\mathbb{P}}(1-p^{-s})=1\Rightarrow\boxed{\zeta(s)=\prod_{p\in\mathbb{P}}(1-p^{-s})^{-1}}$$
    Converting to sum,
    $$\ln\zeta(s)=-\sum_{p\in\mathbb{P}}\ln(1-p^{-s})=\sum_{p\in\mathbb{P}}\sum_{k=1}^{\infty}\frac{(p^{-s})^{k}}{k}=\sum_{k=1}^{\infty}\frac{1}{k}\sum_{p\in\mathbb{P}}\frac{1}{p^{sk}}$$
    As $s\rightarrow 1$ ($\zeta(1)$ is divergent),
    $$\Rightarrow=\sum_{p\in\mathbb{P}}\frac{1}{p^{s}}+\frac{1}{2}\sum_{p\in\mathbb{P}}\frac{1}{p^{2s}}+\cdots$$
    The terms after $1s$ are bounded by $zeta$ and are convergent. But, the first term must be divergent. (These terms are called \textbf{prime zeta functions})
    $$\boxed{\therefore\sum_{p\in\mathbb{P}}\frac{1}{p^{s}}\text{ diverges}}$$
    This shows that not only does the harmonic series diverge, the sum of the reciprocals of primes diverges. Since $zeta(1)$ diverges like $\ln N$,
    $$\sum_{p\in\mathbb{P}}\frac{1}{p^{s}}\text{ diverges like }\ln\ln N$$
    This is smaller than the glacial series $\sum_{k=2}^{N}\frac{1}{k\ln k}$ which gives insight into the denisty of primes. This leads to \textbf{Reimann's prime number theorem}.

\section{Asymptotic Expansion}
If we are interested in $f(x)=e^{x}\int_{x}^{\infty}\frac{e^{-t}}{t}dt$ for large values of $x.$
Repeated integration by parts in a general form gives 
$$=\sum_{k=0}^{n}\frac{(-1)^{k}k!}{x^{k+1}}+(-1)^{n+1}(n+1)!e^{x}\int_{x}^{\infty}\frac{e^{-t}}{t^{n+2}}dt$$
Even though this diverges, it is still useful and can be used as an approximation.
$$f(x)\approx A(x)=\sum_{k=0}^{n}\frac{(-1)^{k}k!}{x^{k+1}} \text{ for large values of } n$$
Asymptotic expansions have two properties. For fixed $x$, arbitrarily large $n$ causes divergence. For a fixed $n$,
arbitrarily large $x$ can get an arbitrarily close approximation.

\subsection{An Asymptotic Expression}
Find an asymptotic expression for
$$f(n)=\int_{0}^{\infty}n^{-x}x^{n}dx$$
Using exponents,
$$=\int_{0}^{\infty}e^{-x\ln n + n\ln x}dx$$
Expanding,
$$g(x)=-x\ln n + n\ln x$$
$$g'(x)=-\ln n + \frac{n}{x}$$
$$g''(x)=-\frac{n}{x^{2}}$$
Setting $g'(x)=0$, we get $x=\frac{n}{\ln n}$.
Now taylor expanding using this value of x as a center,
$$g(x)=-n+n\ln\frac{n}{\ln n}-\frac{1}{2}\frac{n}{(\frac{n}{\ln n})^{2}}(x-\frac{n}{\ln n})^{2}+\cdots$$ 
Since we are working with large values of $x$, we can truncate after the quadratic term.
$$f(n)=\int_{0}^{\infty}e^{-n+n\ln\frac{n}{\ln n}-\frac{1}{2}\frac{n}{(\frac{n}{\ln n})^{2}}(x-\frac{n}{\ln n})^{2}}dx$$
$$=e^{-n+n\ln\frac{n}{\ln n}}\int_{0}^{\infty}e^{-\frac{\ln^{2}n}{2n}(x-\frac{n}{\ln n})^{2}}dx$$
Note that this is very close to a guassian integral with a bound at $0$ instead of $-\infty$. Since the integral
from $-\infty$ to $0$ is very small, we can use a guassian integral.
$$\approx e^{-n+n\ln\frac{n}{\ln n}}\sqrt{\frac{2\pi n}{\ln^{2}n}}$$

\subsection{Another One}
$$f(n)=\int_{0}^{\infty}e^{-x^{2}}x^{nx}dx$$
$$g(x)=-x^{2}+nx\ln x$$
$$g'(x)=-2x+n+n\ln x$$
$$g''(x)=-2+\frac{n}{x}$$
Let $g'(x_{n})=0$ as the equation is transcendental.
$$g(x)=-x_{n}^{2}+nx_{n}\ln x_{n}-\frac{1}{2}(2-\frac{n}{x_{n}})(x-x_{n})^{2}$$
Using a substitution with the first derivative equation,
$$g(x)=x_{n}^{2} -nx_{n}-\frac{1}{2}(2-\frac{n}{x_{n}})(x-x_{n})^{2}$$
Plugging into the integral,
$$f(n)\approx e^{x_{n}^{2}-nx_{n}}\sqrt{\frac{2\pi}{2-\frac{n}{x_{n}}}}$$

\section{Density of Primes}
Previously, we stated that $$\sum_{\text{primes p}}^{N}\frac{1}{p}\text{ diverges like }\ln\ln N$$
So, it behaves like the glacial series.
$$\sum\frac{1}{k\ln k}$$
This shows that the prime density is asymptotic to $\frac{1}{\ln k}$.
The number of primes between $N$ and $N+\Delta N$ is asymptotically $\frac{\Delta N}{\ln N}$.
This gives a good approximation of the \textbf{prime counting function} $\pi (n)=$ the number of primes $\leq n$.
$$\pi(n)=\int_{2}^{n}\frac{dx}{\ln x}$$
This integral is known as the logarithmic integral. Skews determined that the first such crossing must occurr before
$10^{10^{10^{34}}}$.

\section{Zeta Reflection Identity}
We know that $$\int_{0}^{\infty}\frac{x^{s-1}dx}{e^{x}-1}=\Gamma(s)\zeta(s)$$
Writing this as a contour integral in the complex plane, this can be manipulated to achieve the following,
$$\zeta(s)=2^{s}\pi^{s-1}\sin(\frac{\pi s}{2})\Gamma(1-s)\zeta(1-s)$$

\subsection{Positive $s$}
The reflection identity is defined for all complex $s$, but is awkward to use if positive.
Consider $s=2$. $\zeta(2)=\frac{\pi^{2}}{6}$. Due to the zero when pluggin directly, consider $s=2+\epsilon$.
$$\zeta(2+\epsilon)=2^{2+\epsilon}\pi^{1+\epsilon}\sin(\frac{\pi(2+\epsilon)}{2})\Gamma(1-(2+\epsilon))\zeta(1-(2+\epsilon))$$
$$=2^{2}\cdot\pi\cdot 2^{\epsilon}\pi^{\epsilon}\sin(\pi+\frac{\pi\epsilon}{2})\Gamma(-1-\epsilon)\zeta(-1-\epsilon)$$
$$=-4\pi\cdot 2^{\epsilon}\pi^{\epsilon}\sin(\frac{\pi\epsilon}{2})\frac{\Gamma(1-\epsilon)}{(-1-\epsilon)(-\epsilon)}\zeta(-1-\epsilon)$$
See that as $\epsilon\rightarrow 0$ and using the small angle apporximation,
$$=-4\pi\cdot 2^{\epsilon}\pi^{\epsilon}(\frac{\pi\epsilon}{2})\frac{\Gamma(1-\epsilon)}{(-1-\epsilon)(-\epsilon)}\zeta(-1-\epsilon)$$
Cancelling,
$$\zeta(2)=4\pi\cdot\frac{-\pi}{2}\zeta(-1)=-2\pi^{2}\zeta(-1)=-2\pi^{2}\cdot\frac{\pi^{2}}{12}=\frac{\pi^{2}}{6}$$

\subsection{$s=0$}
$$\eta(s)=\sum_{k=1}^{\infty}\frac{(-1)^{k+1}}{k^{s}}=(1-2^{1-s})\zeta(s)$$
Rewriting,
$$\zeta(s)=\frac{\eta(s)}{1-e^{(1-s)\ln 2}}$$
Expanding the exponential and cancelling,
$$=\frac{\eta(s)}{(s-1)\ln 2 - \frac{1}{2}(s-1)^{2}\ln^{2} 2 + \cdots}$$
Since $\eta(1)=1-\frac{1}{2}+\frac{1}{3}-\cdots=\ln 2$, as $s\rightarrow 1$,
$$\zeta(s)\rightarrow\frac{1}{s-1}$$
So, $\zeta$ is defined everywhere except $s=1$. Now, using this result to find $zeta(0)$, we
use a limiting process with the reflection identity.
$$\zeta(\epsilon)=2^{\epsilon}\pi^{\epsilon-1}\sin(\frac{\pi\epsilon}{2})\Gamma(1-\epsilon)\zeta(1-\epsilon)$$
$$\rightarrow 2^{\epsilon}\pi^{\epsilon-1}\sin(\frac{\pi\epsilon}{2})\Gamma(1-\epsilon)\frac{1}{-\epsilon}$$
$$\rightarrow \frac{1}{\pi}\cdots\frac{\pi}{2}(-1)=-\frac{1}{2}$$
$$\therefore\zeta(0)=-\frac{1}{2}$$

\section{Reimann's Hypothesis}
Note that the definition below for zeta only holds for $\Re(s)>1$.
$$\zeta(s)=\sum_{n=1}^{\infty}\frac{1}{n^{s}}$$
This does not mean that $\zeta$ is not defined, it just means that it isn't equal to the RHS outside of the given condition.
Plugging values, we get $\zeta(-1)=-\frac{1}{12}$ and $\zeta(0)=-\frac{1}{12}$. Notice that all negative evens are zeroes of
the function due to the $\sin(\frac{\pi s}{2})$. These zeroes are known as \textbf{trivial zeroes} of the $\zeta$ function.
Since the $\Gamma$ function doesn't have zeroes, the only non-trivial zeroes must come from the reflection $\zeta$ term.
Since $\zeta(x)>0\forall x > 0$. This means that all non-trivial zeroes lie on a \textbf{critical strip} defined as $0<\Re(s)<1$.
Reiman's hypothesis is that all trivial zeroes lie on $\Re(s)=\frac{1}{2}$ but has not been proven. 

\subsection{Dirichlet Series Relationships}
It is true that the relationships between the dirichlet series were first derived using the series. There is an argument that the relationship
doesn't hold everywhere on the complex plane. But, by analytic continuation, the relationship holds everywhere.

\section{Using Sterling Approximations}
Using $n!\approx n^{n}e^{-n}\sqrt{2\pi n}$ for large $n$, we can simplify $\sum_{n=0}^{\infty}\frac{n!^{2}}{(2n)!}x^{n}$
to $\frac{(\frac{x}{4})^{n}}{\sqrt{\pi n}}$ showing that the interval of convergence is $x\in [-4,4)$.

\section{Pendulums}
Find the period of a pendulum of length $L$ with maximum angle $\theta_{max}$ (not for small $\theta_{max}$). Consider the fixed point of the pendulum
to have a height of $0$.
Using conservation of energy,
$$E=\frac{1}{2}mv^{2}+mg(-L\cos\theta)=mg(-Lcos\theta_{max})$$
$$v^{2}=gL(cos\theta-cos\theta_{max})$$
$$v^{2}=\left(\frac{Ld\theta}{dt}\right)^{2}=L^{2}\omega^{2}=gL(cos\theta-cos\theta_{max})$$
$$dt=\frac{|d\theta|}{\sqrt{\frac{g}{L}(\cos\theta - \cos\theta_{max})}}$$
Energy is known as the first integral of Newton's law of motion. Integrating the above equation is known as the second integral.
$$T=4*\sqrt{\frac{L}{g}}\int_{0}^{\theta_{max}}\frac{d\theta}{\sqrt{\cos\theta-\cos\theta_{max}}}$$
For $\theta_{max}=\frac{\pi}{2}$,
$$T=4\sqrt{\frac{L}{g}}\int_{0}^{\frac{\pi}{2}}\cos^{-\frac{1}{2}}d\theta=4\sqrt{\frac{L}{g}}\frac{\Gamma(\frac{1}{4}\Gamma(\frac{1}{2}))}{2\Gamma(\frac{3}{4})}$$
Simplifying,
$$=2\sqrt{\frac{L}{\pi g}}\frac{\Gamma^{2}(\frac{1}{4})\sqrt{pi}}{\sqrt{2}\pi}=\sqrt{\frac{2L}{\pi g}}\Gamma^{2}\frac{1}{4}$$
We know for small angles,
$$T_{SHM}=2\pi\sqrt{\frac{L}{g}}$$
Notice that the period is longer than the one given by simple harmonic motion. However, we still want a general function for the period.
Using the half angle, $\cos\theta=1-2\sin^{2}\frac{\theta}{2}$,
$$T=4*\sqrt{\frac{L}{g}}\int_{0}^{\theta_{max}}\frac{d\theta}{\sqrt{2(\sin^{2}\frac{\theta_{max}}{2}-\sin^{2}\frac{\theta_{max}}{2})}}$$
Using $\sin\frac{\theta}{2}=\sin\frac{\theta_{max}}{2}\cdot\sin\phi$ for substitution to resolve the integral bound,
$$T=2\sqrt{2}\sqrt{\frac{L}{g}}\int_{0}^{\frac{\pi}{2}}\frac{2\sin\frac{\theta_{max}}{2}\cos\phi d\phi}{\cos\frac{\theta}{2}\sqrt{\sin^{2}\frac{\theta_{max}}{2}-\sin^{2}\frac{\theta_{max}}{2}\sin^{2}\phi}}$$
$$=4\sqrt{2}\sqrt{\frac{L}{g}}\int_{0}^{\frac{\pi}{2}}\frac{\sin\frac{\theta_{max}}{2}\cos\phi d\phi}{\cos\frac{\theta}{2}\sin\frac{\theta_{max}}{2}\cos\phi}$$
$$=4\sqrt{2}\sqrt{\frac{L}{g}}\int_{0}^{\frac{\pi}{2}}(1-\sin^{2}\frac{\theta_{max}}{2}\sin^{2}\phi)^{-\frac{1}{2}}d\phi$$
Using the binomial expansion,
$$=4\sqrt{2}\sqrt{\frac{L}{g}}\sum_{k=0}^{\infty}{-\frac{1}{2}\choose k}\sin^{2k}\frac{\theta_{max}}{2}\int_{0}^{\frac{\pi}{2}}\sin^{2k}\phi d\phi$$
$$=4\sqrt{2}\sqrt{\frac{L}{g}}\sum_{k=0}^{\infty}{-\frac{1}{2}\choose k}(-1)^{k}\left(\sin^{2}\frac{\theta_{max}}{2}\right)^{k}\cdot\frac{\Gamma(k+\frac{1}{2})\Gamma(\frac{1}{2})}{2\Gamma(k+1)}$$
$$=2\sqrt{2}\sqrt{\frac{L}{g}}\sum_{k=0}^{\infty}\frac{(-1)^{k}\Gamma(k+\frac{1}{2})}{k!\Gamma(\frac{1}{2})}\cdot(-1)^{k}(\sin^{2}\frac{\theta_{max}}{2})^{k}\cdot\frac{\Gamma(k+\frac{1}{2})\Gamma(\frac{1}{2})}{k!}$$
Cancelling and using Legendre's Duplication Identity,
$$=2\sqrt{2}\sqrt{\frac{L}{g}}\sum_{k=0}^{\infty}\frac{(2^{-2k}\sqrt{\pi}\frac{\Gamma(1+2k)}{\Gamma(1+k)})^{2}}{k!^{2}}(\sin^{2}\frac{\theta_{max}}{2})^{k}$$
$$=2\pi\sqrt{\frac{L}{g}}\sum_{k=0}^{\infty}\frac{(2k)!^{2}}{2^{4k}\cdot k!^{4}}(\sin^{2}\frac{\theta_{max}}{2})^{k}$$
Notice how so many results were used to solve this one problem.

\end{document}

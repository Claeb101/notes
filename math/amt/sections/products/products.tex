\documentclass[../main.tex]{subfiles}
\begin{document}
\chapter{Products}
\section{Definition}
    Let $P_{0}=1$, $P_{k+1}=(1+a_{k+1})P_{k}$. Given $\{a_{k}\}_{k=1}^{\infty}$, this defines a sequence of products $\{P_{k}\}_{k=1}^{\infty}$.
    Adding $1$ in the definition, makes it easy to show convergence as the multiplicitive term needs to tend to 1 or $a_{k}$ needs to tend to $0$.
    This sequence of infinite products is said to converge if $\lim_{k\rightarrow\infty}P_{k}$ exists and is \textbf{nonzero}.
    $\prod_{k=1}^{\infty}(1+a_{k})$ denotes this infinited product.

\section{Convergence}
    The product can be turned into a sum using $\ln$.
    $\prod_{k=1}^{\infty}(1+a_{k})$ converges whenever $\sum_{k=1}^{\infty}a_{k}$ converges \textbf{absolutely}.


\section{Representing a Polynomial}   
    Suppose a polynomial $p(x)$ has roots $-1, 3, 5, \text{and } 12$ each of multiplicity 1 with no other roots. And suppose $p(0)=17$.
    Then,
    $$p(x)=(x+1)(x-3)(x-5)(x-12)\cdot\frac{17}{(1)(-3)(-5)(-12)}=17(1+x)(1-\frac{x}{3})(1-\frac{x}{5})(1-\frac{x}{12})$$
    This form is very important.

\end{document}
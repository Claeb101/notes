\documentclass[../main.tex]{subfiles}
\begin{document}

\chapter{Integration}
\section{Heat Equation}
The heat flow is proportional to the cross sectional area and is inversely 
proportional to the length. The thermal conductivity is this constant of proportionality.
$$\text{heat flow }=\kappa\frac{A(T_{h}-T_{c})}{L}=\kappa A\frac{\partial T}{\partial x}$$
$$\text{net heatflow into region }=\kappa A\left[\frac{T_{2}-T}{\Delta x}-\frac{T-T_{1}}{\Delta x}\right]\Rightarrow\kappa A\frac{\partial^{2}T}{\partial x}L$$
$$=\kappa\frac{\partial^{2}T}{\partial x^{2}} V$$
From chemistry,
$$Q=mc_{p}\Delta T$$
So,
$$\kappa\frac{\partial^{2}T}{\partial x} V=mc_{p}\frac{\partial T}{\partial t}$$
Rearranging,
$$\frac{\partial T}{\partial t}=\frac{\kappa}{\rho c_{p}}\left(\frac{\partial^{2}T}{\partial x^{2}}+\frac{\partial^{2}T}{\partial y^{2}}\right)$$
This gives us the heat equation:
$$\boxed{\frac{\partial T}{\partial t}=\frac{\kappa}{\rho c_{p}}\nabla^{2}T}$$
In steady-state or equilibrium temperature distributions, temperature is time independent so $\nabla^{2}T=0$. 
Thus, steady-state temperature distributions are harmonic functions.

\section{Contour Integrals}
Given a function $f$ and a contour $C$, define the contour integral $\int_{C}f(z)dz$ by parameterizing
$C:z(0)=$ starting point of $C$, $z(1)=$ ending point of C. Then,
$$\int_{C}f(z)dz=\lim_{n\rightarrow\infty}\sum_{k=1}^{n}f\left(z\left(\frac{k}{n}\right)\right)\left(z\left(\frac{k}{n}\right)-z\left(\frac{k-1}{n}\right)\right)$$

\section{Integral Bounds}
If $|f(z)|<M\,\forall\, z\in C$, then
$$|\int_{C}f(z)dz|\leq \lim_{n\rightarrow\infty}\sum_{k=1}^{n}\left|f\left(z\left(\frac{k}{n}\right)\right)\right|\left|z\left(\frac{k}{n}\right)-z\left(\frac{k-1}{n}\right)\right|$$
$$<M\cdot\lim_{n\rightarrow\infty}\sum_{k=1}^{n}\left|z\left(\frac{k}{n}\right)-z\left(\frac{k-1}{n}\right)\right|\leq ML$$

\section{Fundamental Theorem of Contour Integrals}
\subsection{Definition}
If $F(z)$ can be found for which $F'(z)=f(z)\,\forall\, z\in \mathbb{C}$,
then, $$\int_{c}f(z)dz=F(z_{end})-F(z_{start})$$
\subsection{Proof}
Parameterize $C:z(t),t\in[0,1]$. Let $z_{k}=z\left(\frac{k}{n}\right)$
$$F(z(1))-F(z(0))=F(z_{n})-F(z_{0})=F(z_{n})-F(z_{1})+F(z_{1})-F_{z_{0}}$$
$$=\sum_{k=1}^{n}(F_{z_{k}}-F_{z_{k-1}})=\sum_{k=1}^{n}(F(z_{k-1})+F'(z_{k-1})(z_{k}-z_{k-1})+\xi(z_{k},z_{k-1})-F(z_{k-1}))$$
$$=\sum_{k=1}^{n}F'(z_{k-1})(z_{k}-z_{k-1})+\sum_{k=1}^{n}\xi(z_{k},z_{k-1})$$
As $n\rightarrow\infty$, we get the Fundamental Theorem of Contour Integerals.

\subsection{Corollary}
If $F'(z)=f(z)\,\forall\, z\in D$ where $D$ is a domain, then
$\int_{c}f(z)dz$ is independent of path for every controur $C\subset D$.
\subsection{Branch Cuts}
$$\int_{C}\frac{dz}{z^{2}+1}=\int_{C}\left(\frac{1}{z-i}-\frac{1}{z+i}\right)\frac{dz}{2i}$$
If C avoids the following branch cuts, this is valid:
$$=\frac{1}{2i}\left(\log (z-i)-\log (z+i)\right)_{\text{start}}^{\text{end}}$$
If C crosses the branch cuts, this no longer works. We just need to manipulate the branch cuts
so that the function does not cross it. So if $C$ crosses the top branch cut in a downwards facing parabola,
we can do the following to move the top branch cut.
$$=\frac{1}{2i}\left(\log (-i(z-i))+\frac{i\pi}{2}-\log (z+i)\right)_{\text{start}}^{\text{end}}$$

\section{Path Independence}
If $C_{1},C_{2}$ are on $C$,
$$\int_{C_{1}}f(z)dz-\int_{C_{2}}f(z)dz=\oint_{C}f(z)dz=\oint_{c}((udx-vdy)+i(vdx+udy))$$
Using Green's theorem (not rigorous),
$$=\iint_{D}((-v_{x}-u_{y}+i(u_{x}-v_{y}))dA$$
This $=0$ if $f(z)$ satisfies Cauchy reimann. So, whenever $f(z)$ is analytic throughout a region that allows
one contour to be deformed to the other.

\section{Contour Deformation}
If $C_{1}$ and $C_{2}$ both begin at $z_{1}$ and end at $z_{2}$ and $C_{1}$ can be continuously deformed into $C_{2}$ 
without leaving the domain $D$ of analyticity of $f(z)$, then
$$\int_{C_{1}}f(z)dz=\int_{C_{2}}f(z)dz$$

\section{Cauchy's Integral Formula}
Suppose $f(z)$ is analytic on a simply connected domain $D$ containing the closed loop $C$ with $z_{0}$ in its interior.
$$f(z_{0})=\frac{1}{2\pi i}\oint_{C}\frac{f(z)}{z-z_{0}}dz$$

\section{Cauchy's Integral Formula for Derivatives}
For analytic $f$ at $z_{0}$.
$$f(z)-f(z_{0})=\frac{1}{2\pi i}\oint_{C}\frac{f(\xi)}{\xi -z}d\xi - \frac{1}{2\pi i}\oint \frac{f(\xi)}{\xi-z_{0}}d\xi$$
$$=\frac{1}{2\pi i}\oint_{C}f(\xi)\frac{z-z_{0}}{(\xi-z)(\xi-z_{0})}d\xi$$
Manipulating,
$$\frac{f(z)-f(z_{0})}{z-z_{0}}=\frac{1}{2\pi i}\oint_{C}\frac{f(\xi)}{(\xi-z)(\xi-z_{0})}d\xi$$
Taking limits as $z\rightarrow z_{0}$,
$$f'(z_{0})=\frac{1}{2\pi i}\oint_{C}\frac{f(\xi)}{(\xi-z)^{2}}d\xi$$
This shows that for a function that is analytic at $z_{0}$, it's derivative also exists. This differentiation can continue yeilding the following:
$$f^{(k)}(z_{0})=\frac{k!}{2\pi i}\oint_{C}\frac{f(\xi)}{(\xi-z)^{k+1}}d\xi$$.
Thus a function that is analytic at a given point is infinitely differentiable at that given point.

\section{Liouville's Theorem}
\subsection{Statement}
A bounded entire function is constant.

\subsection{Proof}
$$f'(z_{0})=\frac{1}{2\pi i}\oint_{|z-z_{0}|=R}\frac{f(\xi)}{(\xi-z)^{2}}d\xi$$
for an arbitrary point $z_{0}$ by CIFFD.
Using integral bounds, suppose
$$|f(z)|<M \,\forall\, z\in\mathbb{C}$$
then,
$$|f'(z)|<\frac{1}{2\pi}\frac{M}{R^{2}}2\pi R=\frac{M}{R}$$
As $R\rightarrow\infty$, this statement still holds true so $f'(z)=0$ so $f(z)$ is constant.

\section{Another Statement}
Suppose $f(z)$ is entire and $|f(z)|<10|z|^{3}\sqrt{|z|}\,\forall\, z:|z|>30$.
Then $f(z)$ is a polynomial of degree 3 at most.

\subsection{Proof}
For any point $z_{0}$,
$$f^{(4)}(z_{0})=\frac{4!}{2\pi i}\oint_{|z-z_{0}|=R}\frac{f(\xi)}{(\xi-z_{0})^{5}}d\xi$$
is true by CIFFD. For $R > 30+|z_{0}|$,
$$|f^{(4)}(z_{0})|<\frac{24}{2\pi}\cdot\frac{10R^{\frac{7}{2}}}{R^{5}}\cdot 2\pi R=\frac{240}{R^{\frac{1}{2}}}$$
Expanding $R$ without bound, this must still be true to $f^{(4)}(z)=0$.

\subsection{General Statemnt}
If a entire function is bounded by any power of $z$, it must be a polynomial function or constant. Any non-polynomial 
entire functions must grow faster than any power of $z$ (think exponential).

\section{Poisson's Formula}
$$f(z)=\frac{1}{2\pi i}\oint_{|\xi|=R}\frac{f(\xi)}{\xi-z}d\xi$$
Adding a term,
$$f(z)=\frac{1}{2\pi i}\oint_{|\xi|=R}\frac{f(\xi)}{\xi-z}d\xi+\frac{1}{2\pi i}\oint_{|\xi|=R}\frac{f(\xi)\overline{z}}{R^{2}-\xi\overline{z}}d\xi$$
Note that the modulus of the singularity is $|\xi|=|\frac{R^{2}}{z}|=\frac{R^{2}}{z}>R$ for $|z|<R$. This means that the added term is $0$.
$$=\frac{1}{2\pi i}\oint_{|\xi|=R}f(\xi)\left(\frac{1}{\xi -z}+\frac{\overline{z}}{R^{2}-\xi\overline{z}}\right)d\xi$$
Note that when this fraction is combined, the numerator is independent of $z$. 
$$=\frac{1}{2\pi i}\oint_{|\xi|=R}\frac{R^{2}-r^{2}}{(Re^{it}-z)(R^{2}-Re^{it}z)}$$
$$=\frac{1}{2\pi i}\oint_{|\xi|=R}\frac{R^{2}-r^{2}}{Re^{it}}\cdot\frac{1}{(Re^{it}-z)(Re^{-it}-\overline{z})}$$
$$=\frac{R^{2}-r^{2}}{2\pi i}\oint_{|\xi|=R}\frac{f(\xi)}{Re^{it}|Re^{it}-z|^{2}}$$
Completing the substitution of $\xi=Re^{it}$,
$$=\frac{R^{2}-r^{2}}{2\pi i}\int_{0}^{2\pi}\frac{f(Re^{it})}{Re^{it}|Re^{it}-z|^{2}}\cdot iRe^{it}dt$$
Simplifying,
$$=\frac{R^{2}-r^{2}}{2\pi}\int_{0}^{2\pi}\frac{f(Re^{it})}{|Re^{it}-z|^{2}}dt$$
Taking the real part,
$$u(x,y)=\frac{R^{2}-r^{2}}{2\pi}\int_{0}^{2\pi}\frac{u(R\cos t, R\sin t)}{|Re^{it}-z|^{2}}$$
This shows that if we know the value of a harmonic function everywhere on circle, we can use this to find information of its values inside.

\section{Harnack's Inequality}
Manipulating Poisson's formula above,
$$u(r\cos \theta, r\sin\theta)=\frac{R^{2}-r^{2}}{2\pi}\int_{0}^{2\pi}\frac{u(R\cos t, R\sin t)}{R^{2}-2Rr\cos(t-\theta)+r^{2}}dt$$
Taking bounds,
$$u(r\cos \theta, r\sin\theta)\leq\frac{R^{2}-r^{2}}{2\pi}\int_{0}^{2\pi}\frac{u(R\cos t, R\sin t)}{(R-r)^{2}}dt$$
$$=\frac{R+r}{R-r}\frac{1}{2\pi}\int_{0}^{2\pi}u(R\cos t, R\sin t)dt=\frac{R+r}{R-r}u(0,0)$$
Using the same manipulation on the other side,
$$
\frac{R-r}{R+r}u(0,0)
\leq u(r\cos \theta, r\sin\theta)
$$
This yields Harnack's Inequality,
$$
\frac{R-r}{R+r}u(0,0)
\leq u(r\cos \theta, r\sin\theta)
\leq \frac{R+r}{R-r}u(0,0)dt
$$

\section{Reimann Mapping Theorem}
Any simply connected domain whose boundary consists of more than two points can be mapped in a one to one invertible analytic way
to the interior of a unit disk.

\end{document}

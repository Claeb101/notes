\documentclass[../main.tex]{subfiles}
\begin{document}

\chapter{Mappings}
\section{Mapping Basics}
\subsection{Visulization of Complex Functions}
    \subsection{Animation}
        Complex functions are usually visualized by graphing the input space and then the output space
        with perhaps an animation transforming a grid on the complex plane. Suppose, in a linear algebra sense,
        that the function operates on the entire domain, transforming it into the codomain.

    \subsubsection{Color Map}
        A color spectrum is used to denote different points on the complex plane. The output is a 2D color plot
        where the color at some point denotes the mapped value of that point through the function.

\subsection{Conformal Map}
    A conformal map is a map that preserves angles and orientations.

\subsection{Anti-conformal Map}
    An anti-conformal map is a map that preserves angles and but reverses orientation.

\section{Functions and "Maps"}
$$f(z)=u+iv$$
\subsection{Tangents}
    Curve of constant y:
        $$u_{x}+iv_{x}$$
    Curve of constant x:
        $$u_{y}+iv_{y}$$

\subsection{Conformal Requirements}
    Confromal maps requires:
        $$u_{y}+iv_{y}=i(u_{x}+iv_{x})$$
    The multiplication by $i$ (rotates by $90^{\circ}$) allows angles and orientation to be preserved.

\subsection{Cauchy-Riemann Equations}
    By equating the imaginary and real parts of the conformal requirement above,
    $$u_{x}=v_{y}$$
    $$u_{y}=-v_{x}$$
    If these equations are satisfied for some function $f$, then $f$ is a conformal map.

\subsection{Substitutions}
    Note the following substitutions,
    $$x=\frac{z+\overline{z}}{2}\text{, }y=\frac{z-\overline{z}}{2i}$$
    These allow conversion of a function from in terms of $x,y\in\mathbb{Z}$ to $z\in\mathbb{C}$.

\subsection{Dependence on $z$ and $\overline{z}$}
    For $f=u+iv$,
    $$\frac{\partial f}{\partial z}=\frac{\partial f}{\partial x}\cdot\frac{\partial x}{\partial z}+\frac{\partial f}{\partial y}\cdot\frac{\partial y}{\partial z}=(u_{x}+iv_{x})\frac{1}{2}+(u_{y}+iv_{y})\frac{1}{2i}=\frac{u_{x}+v_{y}}{2}+i\frac{v_{x}-u_{y}}{2}$$
    $$\frac{\partial f}{\partial \overline{z}}=\frac{\partial f}{\partial x}\cdot\frac{\partial x}{\partial \overline{z}}+\frac{\partial f}{\partial y}\cdot\frac{\partial y}{\partial \overline{z}}=(u_{x}+iv_{x})\frac{1}{2}+(u_{y}+iv_{y})\frac{-1}{2i}=\frac{u_{x}-v_{y}}{2}+i\frac{v_{x}+u_{y}}{2}$$
    $\therefore$ if Cauchy-Riemann is satisfied, $f$ is only in terms of $z$ as $\frac{\partial f}{\partial \overline{z}}=0$.


\section{Stereographic Projection}

\subsection{1D-2D Case}
    Consider mapping the real axis to a unit circle. For each point on the real axis, draw a line segment to the top of the circle $i$ and map it to the point of intersection of that extended linesegment with the circle.
    Note that this works from the real axis to the circle, but this does not work from the circle to the real axis.
    The circle is known as \textbf{compact} (topology) while the real axis is not.
    But, notice that every point except the top of the circle is mapped to. Thus, by adding the \textbf{point of infinity} at $i$ which allows for a 2-way mapping.

\subsection{2D-3D Case}
    In the same way, the entire complex plane can be mapped to a unit where the point at (1, 0, 0) is the point at infinity. Since by adding one point, the plane can be compactified, this is known as \textbf{one point compactification}.
    Note that $i\infty$, $(1-\sqrt{3})\infty$, etc. all map to the same point of infinity. Just how the origin has no argument, the point at $\infty$ has no argument. This sphere is known as the \textbf{Reimann Sphere}.

\section{Inversion}
\subsection{Function}
    The inversion function is defined as follows:
    $$f(z)=\frac{1}{z}$$
    $$f(z)=\frac{1}{x+iy}=\frac{x}{x^{2}+y^{2}}+i\frac{-y}{x^{2}+y^{2}}$$
    Curves of constant y has a tangent "vector":
    $$\frac{\partial f}{\partial x}=\frac{y^{2}-x^{2}}{(x^{2}+y^{2})^{2}}+i\frac{2xy}{(x^{2}+y^{2})^{2}}$$
    Curves of constant x has tangent "vector":
    $$\frac{\partial f}{\partial y}=\frac{-2xy}{(x^{2}+y^{2})^{2}}+i\frac{y^{2}-x^{2}}{(x^{2}+y{2})^{2}}$$
    Notice $i\frac{\partial f}{\partial y}=\frac{\partial f}{\partial x}$.

\subsection{Mappings}
    Inversion turns "circle/lines" into "circle/lines". Circles remain finite and don't go to infinity while lines all do. 
    Under inversion, $0^{\pm}\rightarrow\pm\infty$ and $\pm\infty\rightarrow0^{\pm}$. Inversion maps points with arbitrarily large modulus' to points with arbitrarily small modulus' (near origin).
    
    \subsubsection{Circles}
        Circle's are determined by any 3 non-collinear points (one-to-one). Circles that pass through the origin are mapped to lines not passing through the origin.

    \subsection{Lines}
        If a line doesn't pass through the origin, under inversion, the line is mapped to a circle passing through the origin and the other 2 mapped points defining the line.

\section{Mobius Transformation}
    A mobius transformation is a function of the form $$f(z)=\frac{az+b}{cz+d}\mid ad-bc\neq 0$$
    This is the most general transformation that maps the whole Riemann sphere (including the point at infinity) to itself in a confromal and one-to-one manner.
    \subsection{Inversion and Identity}
        Note that the inversion mapping is simply when $a=0,\, b=1,\, c=1,\, d=0$. And the identity tranformation is when $a=z,\, b=0,\, c=0,\, d=1$
    \subsection{$c=0$}
        If $c=0$ this is just a dilation and a translation. In this case, circles and lines are kept separate and "do not mix".
    \subsection{Composition}
        Begin with:
        $$z$$
        Rotate/dilate by $c$:
        $$cz$$
        Translate by $d$:
        $$cz+d$$
        Invert:
        $$\frac{1}{cz+d}$$
        Rotate/dilate by $\frac{bc-ad}{c}$:
        $$\frac{bc-ad}{c(cz+d)}$$
        Translate by $\frac{a}{c}$:
        $$\frac{bc-ad}{c(az+d)}+\frac{a}{c}$$
        Simplify:
        $$\frac{az+b}{cz+d}$$
        This shows that any mobius transformation is simply a composition of conformal, one-to-one, circle/line preserving functions. So, a mobius transformation must be 
        one-to-one and conformal while preserving circle/line.

    \subsection{Finding a transformation}
        Mobius transformations are defined by 3 unique maps.
        Find a Mobious transformation that maps $2$ to $0$, $i$ to $\infty$, and $1+i$ to $1$. 
        $$\frac{z-2}{z-i}\cdot\frac{1+i-i}{1+i-2}\cdot 1 = -\frac{1+i}{2}\frac{z-2}{z-i}$$
        Does the line $2x+y=5$ map to a circle or a line? => Maps to a circle.
        What about the circle $x^{2}+y^{2}+2y=3$? => Maps to a line.

    \subsection{Another Example}
        Find a mobius transformation that maps the exterior of the circle $|z|=1$ to the region above the line $x+y=1$.
        We can start by mapping the circular boundary to the line. Picking 3 points (1, -1, i) on the circle and determining orientation (clock-wise),
        we can choose the points that the 3 points map to. 1 maps to i, -1 maps to 1, and i maps to $\infty$.
        Since i goes to $\infty$, the transformation is of the form,
        $$\frac{az+b}{z-i}$$
        Pluggin in the two mappings we get,
        $$\frac{a+b}{1-i}=i$$
        $$\frac{-a+b}{-1-i}=1$$
        Solving the equations,
        $$b=0\text{, } a=1+i$$
        $$\therefore f(z)=\frac{(1+i)z}{z-i}$$
        This checks out using the origin as a test point, so this is true.
        Note that there are an infinite number of mobius transformations that do this mapping.
    
\end{document}
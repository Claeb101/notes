\documentclass[../main.tex]{subfiles}
\begin{document}

\chapter{The Basics of Complex Numbers}
\section{Complex Numbers Review}
\subsection{Definition}
    Complex numbers are of the form $a+bi$, $a,b\in\mathbb{R}$ and $i^{2}=-1$.

\subsection{Complex Conjugate}
    \textbf{Definition}: $\overline{a+bi}=a-bi$
    If $z=a+bi$, then $z\cdot\overline{z}=a^{2}+b^{2} > 0$ unless $z=0$.

\subsection{Multiplication}
    The multiplication of 2 complex numbers is a complex number. Multiplication proceeds via
    basic algebraic distribution.

\subsection{Division}
    Division of complex numbers is defined as long as the denominator is not 0.
    To do division operations, multiple the top and bottom by the complex conjugate of the 
    denominator and simplify.

\subsection{Modulus}
    \textbf{Modulus} of $z=a+ib$ is given by
    $$|z|=\sqrt{a^{2}+b^{2}}$$
    $$|z_{1}z_{2}|^{2}=z_{1}z_{2}\overline{z_{1}}\overline{z_{2}}=|z_{1}|^{2}|z_{2}|^{2}$$
    $$|z_{1}z_{2}|=|z_{1}||z_{2}|$$

\section{Geometry of the Complex Plane}
\subsection{The Plane}
    The complex plane is just like the xy-plane except it has a real axis "x" and a complex axis "y".

\subsection{Distances}
    $|z|$ gives the distance from $z$ to the origin.
    $|z_{2}-z_{1}|$ gives the distance from $z_{2}$ to $z_{1}$ in the complex plane.

\subsection{Polar Representation}
    $r=|z|$ and $\theta=\arg(z)$ "the argument of z". We use the "normal" version.
    $(1, \pi) \text{ or } (1, 3\pi) \text{ instead of } (-1, -\pi)$
    
    \subsubsection{Principle Argument:}
        $$-\pi\leq Arg(z)\leq\pi$$
    
    \subsubsection{rcis form: }
        $a=rcos\theta, b=rsin\theta, z=r(cos\theta+isin\theta)$
    
    \subsubsection{Conjugates}
        If $\theta=Arg(z)$, $\overline{rcis\theta}=rcis(-\theta)$

    \subsubsection{Multiplication}
        Suppose $z_{1}=r_{1}cis\theta_{1}$ and $z_{2}=r_{2}cis\theta_{2}$,
        $$z_{1}z_{2}=r_{1}r_{2}cis(\theta_{1}+\theta_{2})$$

        $$arg(z_{1}z_{2})=arg(z_{1})+arg(z_{2})$$
        \textbf{Conclusion: } Multiplication in the complex plane consists of a rotation and a dilation.

\subsection{Exponential Representation + Taylor Series}
    $$cis\theta=e^{i\theta}$$
    $$e^{i\theta}=1+i\theta+\frac{(i\theta)^{2}}{2!}+\frac{(i\theta)^{3}}{3!}+\cdots=(1-\frac{\theta^{2}}{2!}+\frac{\theta^{4}}{4!}+\cdots)+i(\theta-\frac{\theta^{3}}{3!}+\frac{\theta^{5}}{5!}\cdots)$$
    $$e^{i\theta}=cos\theta+isin\theta$$
    $$cos\theta=\frac{e^{i\theta}+e^{-i\theta}}{2}$$
    $$sin\theta=\frac{e^{i\theta}-e^{-i\theta}}{2i}$$
    $$cosz=\frac{e^{iz}+e^{-iz}}{2}$$
    Cosine can be used on the entire complex plane.
\subsection{Hyperbolic Functions}
    $$\cos(a+ib)=\frac{e^{i(a+ib)-e^{-i(a+ib)}}}{2}=\frac{e^{ia}e^{-b}+e^{-ia}e^{b}}{2}$$
    $$=\cos a\cdot\frac{e^{b}+e^{-b}}{2}-i\sin a\cdot\frac{e^{b}-e^{-b}}{2}$$
    $$\cosh z=\frac{e^{z}+e^{-z}}{2}$$
    $$\sinh z=\frac{e^{z}-e^{-z}}{2}$$
    $$\tanh z = \frac{e^{z}-e^{-z}}{e^{z}+e^{-z}}$$

\section{Exponentiation}
Converting to exponential form is a lot easier than using binomial theorem.
Multiplication, division, and exponentiation is a lot easier in exponential notation.
$$(1+i\sqrt{3})^{7}=(2e^{i\frac{\pi}{3}})^7=2^{7}\cdot e^{i\frac{7\pi}{3}}$$
\subsection{Not Bijective}
    $$\forall a\in\mathbb{R} : a\neq 0\text{, }e^{z}=a\text{ for inifinitely many z.} $$
\subsection{Fractional Exponentiation}
    Using a exponential notation, $\pm$ results arise for different arguments.
    \subsubsection{Principal Value}
        Convention is to use the principal argument during calculations. Remeber that
        the principal argument is \textbf{not always positive} but the one between $-\pi$ and $\pi$.
        \\\\
        \textbf{Example: }   
            $$\text{For } n \in \mathbb{Z}\text{, }i=e^{i\frac{\pi}{2}}\cdot e^{i\cdot 2\pi n}\Rightarrow i^{\frac{1}{3}}=\text{cis}\left(\frac{\pi}{6}+\frac{2\pi n}{3}\right)$$
            Note that these values form an equilateral triangle when polygon. Note that $i^{\frac{1}{q}}$ for $q\in\mathbb{N}$, results in
            an regular $n-gon$.
        \\\\
        \noindent\textbf{Calculators:} TI calculators do not always return the principal value.\\
        \noindent\textbf{Phase:} $e^{i\theta}$ is called a \textbf{phase factor}.

\subsection{Raising to the $i$ Power}
    Raising something to the $i$ power "switches" the argument and modulus.
    $$i^i=(e^{i\frac{\pi}{2}+2in\pi})^{i}=e^{-\frac{\pi}{2}-2n\pi}$$
    $$2^{i}=(e^{\ln2}\cdot e^{2\pi in})^{i}=e^{-2\pi n}\cdot \text{cis}(\ln2)$$

\end{document}
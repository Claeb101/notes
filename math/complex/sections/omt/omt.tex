\documentclass[../main.tex]{subfiles}
\begin{document}

\chapter{The Open Mapping Theorem}
\section{The Open Mapping Theorem}
    \subsection{Statement}
        A \textbf{nonconstant} function analytic on domain D maps D to an open set.
    
    \subsection{Results}
        If $f(z)$ is analytic on domain $D$ and $|f(z)|$ or $Arg(f(z))$ is constant,
        $f(z)$ must be constant.

\section{The Maximum Modulus Principle}
    \subsection{Statement}
    Every function $f(z)$ \textbf{analytic} on domain $D$ attains the maximum modulus
    on every closed connection region $R \in D$ on the boundary of $R$.

    \subsection{Results}
        Suppose $f(z)=g(z)$ are both analytic on the boundary $|z|<2$, then $f(z)=g(z)\,\forall\, z\mid |z|<2$.

\section{Fundamental Theorem of Algebra}
    \subsection{Statement}
    Every nonconstant polynomial $p(z)$ has at least one zero in the complex plane.
    
    \subsection{Proof}
        Assume that polynomial $p(z)$ is never zero in the complex plane. 
        Since all polynomials are entire an $p(z)\neq 0$, $\frac{1}{p(z)}$ is entire.
        So, $\frac{1}{p(z)}$ achieves its maximus modulus on every disk $|z|\leq R$. On $|z|=R$,
        as $R$ grows without bound, $|\frac{1}{p(Re^{it})}|\rightarrow 0$. 
        This is a contradiction. So, $p(z)$ must have at least one zero.

    \subsection{Generalization}
        Given $f(z)$ is entire, if given any $N > 0$ a value of $R$ can be found for which $|f(Re^{it})|>N\,\forall\, t\in\mathbb{R}$.
        We can conclude that $f(z)$ has at least one zero in the complex plane.

\section{A Strong Result}
    \subsection{Definition}
    Suppose nonconstant $u(x,y)$ is harmonic on a simply connected domain $D$. Then, we can find a harmonic conjugate $v(x,y)$
    for which $f(z)=u+iv$ is analytic on D. This can be used in conjuction with the Maximum Modulus Principle to prove
    that $u(x,y)$ can achieve neither a local maximum nor local minimum on $D$.
    \subsection{Proof}
        Consider $|e^{f(z)}|=|e^{u+iv}|=e^{u}$. Then, consider $|e^{-f(z)}|=e^{-u}$. Using MMP, this can now be proved.

\end{document}
